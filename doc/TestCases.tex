\section{Test Cases}
\textbf{Test Case Layout} There are three parts to each test case:
\begin{itemize}
	\item A Test Details section, detailing an overview of the testing methods.
	\item A Scenario Matrix which describes all possible input and output combinations for the use case being tested
	\item A Testing Matrix which describes test cases by their input conditions and expected output
	\item A Test Case to Scenario Table which relates all test cases to the scenarios they represent
\end {itemize}
\subsection{Initiate Web Crawl}
\textbf{Test Details}\\\\
This test will test the crawler's data inputs by simulating different combinations of proper and improper user inputs, specifically the email and site fields.\\\\
\textbf{Scenario Matrix}\\\\
\begin{tabular}{|c|c c|}
	\hline
	Scenario ID & Originating Flow & Alternate Flow  \\ \hline \hline
	S0 & Main &  \\ \hline
	S1 & Main & Alt. \\ \hline
\end{tabular}\\\\\\\\
\textbf{Test Matrix}\\\\
\begin{tabularx}{\textwidth}{|c c c X|}
	\hline
	\textbf{Test Case ID} &\bf{Email Field} & \textbf{Site Field} & \textbf{Expected Result}  \\ \hline
	T0 & Present & Present & System recognizes valid input and begins crawl \\ \hline
	T1 & Present & Missing & System recognizes Site field is missing, notifies user \\ \hline
	T2 & Missing & Present & System recognizes Email field is missing, notifies user  \\ \hline
	T3 & Missing & Missing & System recognizes both fields are missing, notifies user  \\ \hline
\end{tabularx}\\\\\\\\
\textbf{Test Case to Scenario Table}\\\\
\begin{tabular}{|r||c|c|}
	\hline
	& S0 & S1 \\ \hline \hline
	T0 & X & \\ \hline
	T1 & & X \\ \hline
	T2 & & X \\ \hline
	T3 & & X \\ \hline
\end{tabular}


\subsection{Display Trace Log}
\textbf{Test Details}
In order to test logging functions of the crawler, the logging module will be fed mocked inputs from successful processes and unsuccessful processes.\\\\
\textbf{Scenario Matrix}\\\\
\begin{tabular}{|c|c c|}
	\hline
	Scenario ID & Originating Flow & Alternate Flow  \\ \hline \hline
	S0 & Main &  \\ \hline
	S1 & Main & Alt. \\ \hline
\end{tabular}\\\\\\\\
\textbf{Test Matrix}\\\\
\begin{tabularx}{\textwidth}{|c c X|}
	\hline
	\textbf{Test Case ID} & \bf{Successful Process} & \textbf{Expected Result}  \\ \hline
	T0 & Yes & System intakes all necessary logging data from mocked processes and logs it.  There is no failure flag. \\ \hline
	T1 & No & System intakes all necessary logging data from mocked processes and logs it.  There is a failure flag present \\ \hline
\end{tabularx}\\\\\\\\
\textbf{Test Case to Scenario Table}\\\\
\begin{tabular}{|r||c|c|}
	\hline
	& S0 & S1 \\ \hline \hline
	T0 & X & \\ \hline
	T1 & & X \\ \hline
\end{tabular}


\subsection{Compare Previous Crawls}
\textbf{Test Details}
A mock database with mock previous crawl data will be provided to the comparison module as well as mock crawl data.  Comparison data will be generated, and stored in the database.\\\\
\textbf{Scenario Matrix}\\\\
\begin{tabular}{|c|c c|}
	\hline
	Scenario ID & Originating Flow & Alternate Flow  \\ \hline \hline
	S0 & Main &  \\ \hline
	S1 & Main & Alt. \\ \hline
\end{tabular}\\\\\\\\
\textbf{Test Matrix}\\\\
\begin{tabularx}{\textwidth}{|c c X|}
	\hline
	\textbf{Test Case ID} & \bf{Previous Data Present} & \textbf{Expected Result}  \\ \hline
	T0 & Yes & System receives data, and takes in data from the database, performs comparisons and generates report. The results are fed back into the database. \\ \hline
	T1 & No & System receives data, and attempts data retreival from database.  When no data is found, comparisons are skipped and a report is generated.  The results are fed back into the database \\ \hline
\end{tabularx}\\\\\\\\
\textbf{Test Case to Scenario Table}\\\\
\begin{tabular}{|r||c|c|}
	\hline
	& S0 & S1 \\ \hline \hline
	T0 & X & \\ \hline
	T1 & & X \\ \hline
\end{tabular}


\subsection{Report Results}
\textbf{Test Details}
A mock analysis is provided to the reporting module.  The module then attempts to email the report.\\\\
\textbf{Scenario Matrix}\\\\
\begin{tabular}{|c|c c|}
	\hline
	Scenario ID & Originating Flow & Alternate Flow \\ \hline \hline
	S0 & Main & \\ \hline
	S1 & Main & A \\ \hline
	S2 & Main & B \\ \hline
\end{tabular}\\\\\\\\
\textbf{Test Matrix}\\\\
\begin{tabularx}{\textwidth}{|c c c X|}
	\hline
	\textbf{Test Case ID} & \textbf{Email System Available} & \textbf{Email Error} & \textbf{Expected Result}  \\ \hline
	T0 & No & No & The system generates a report and successfully generates an SMTP transaction with the email system \\ \hline
	T1 & Yes & N/A & The system generates a report and attempts an SMTP transaction with the email system.  When the SMTP request times out, the report is stored for sending at a later time. \\ \hline
	T2 & No & Yes & The system generates a report and attempts an SMTP transaction with the email system.  When an error report is received, the report is stored for sending at a later time \\ \hline
\end{tabularx}\\\\\\\\
\textbf{Test Case to Scenario Table}\\\\
\begin{tabular}{|r||c|c|c|}
	\hline
	& S0 & S1 & S2 \\ \hline \hline
	T0 & X & & \\ \hline
	T1 & & X & \\ \hline
	T2 & & & X \\ \hline
\end{tabular}


%\subsection{Test Case Description}
%\begin{itemize}
%    \item Mock the crawling software and present the web interface to the user and verify that the correct variables are passed and the correct functions are called.
%    \item Mock a crawler that creates a log file and make sure it is in the correct location and is able to be read.
%    \item Mock input into a database and make sure that the dates as well as the information acquired is present and easily comparable.
%    \item Mock an e-mail client and make sure that all relevant information is present such as the e-mail from the requester, the website crawled, and the where the data is stored in the database (crawl ID).
%    \item Unit test the retreived file tree against all of the files in the dummy web site.
%    \item Run the crawler on a dummy web site and verify that a log file is produced.
%    \item Unit test the parsed HTML data against the expected data.
%    \item Unit test parsed Javascript data against expected data.
%    \item Verify that each aspect of the website is being stored in the database (using a mock database).
%\end{itemize}

