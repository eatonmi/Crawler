\section{Introduction}

403 Security, a subsidiary of WDD Inc. , is a website security consulting firm based out of Indianapolis.  It’s headed by Alan Wlasuk, our primary contact at the company and the manager of our project. 403 Security has only existed  since 2007, but its employees are extremely experienced in the field of web security. To acquire customers, 403 Security offers free scanning and analysis services to websites; its primary service to clients is fixing the flaws it finds.

The current process of offering a free, low-impact (non-penetration-testing) scan is labor intensive: a developer needs to configure Acunetix, a mid-grade security and penetration testing tool, to scan a website without doing any active penetration testing. Acunetix puts out a very detailed and technical report on what it detected; the report requires significant effort to analyze.

In order to make its marketing less resource-intensive, 403 Security wants our team to create a low-impact security scanner from the ground up, one whose original purpose is to passively scan rather than to actively penetrate websites. There are both legal and financial reasons why this is preferable to sticking with Acunetix.

From a business standpoint, it saves developer-hours to  generating easy-to-read reports rather than highly technical reports requiring additional analysis. Acunetix will give great information regarding HTML and SQL injection, cross-site scripting, and other technical attacks but doesn’t directly say what version of server software or CMS software a website might be operating on. Greater savings could be realized if Acunetix could automatically scan a website when one is submitted rather than requiring a developer or technician to manually process each website and a salesperson to verify the identity of the person requesting the site scan.
    
It would be legally questionable at best to automate penetration tests, or even non-penetration scans with a tool that is difficult to configure and is intended for penetration tests, from anonymous requests over the internet. This would no longer be a problem if our tool were used: it couldn’t do harm to a website or collect information a Google search crawler couldn’t do or collect so there’s no significant chance it could cross legal boundaries.

403 Security will be treating our team as a third-party consulting company, and we will be treating 403 Security as a client of ours. We are very excited to have this opportunity to work with 403 Security on developing such an interesting tool and look forward to the upcoming year.

