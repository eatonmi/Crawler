\section{Executive Summary}
403 Security, a subsidiary of WDD Inc. , is a website security consulting firm based out of Indianapolis.  It’s headed by Alan Wlasuk, our primary contact at the company and the manager of our project. 403 Security has only existed  since 2007, but its employees are extremely experienced in the field of web security. To acquire customers, 403 Security offers free scanning and analysis services to websites; its primary service to clients is fixing the flaws it finds.\\\\
The current process of offering a free, low-impact (non-penetration-testing) scan is labor intensive: a developer needs to configure Acunetix, a mid-grade security and penetration testing tool, to scan a website without doing any active penetration testing. Acunetix puts out a very detailed and technical report on what it detected; the report requires significant effort to analyze.\\\\
In order to make its marketing less resource-intensive, 403 Security wants our team to create a low-impact security scanner from the ground up, one whose original purpose is to passively scan rather than to actively penetrate websites. There are both legal and financial reasons why this is preferable to staying with Acunetix.\\\\
In order to generate new clients in a way that does not require a validation of ownership, saving time and money for 403 Securty, the Security Crawler system has been commissioned.  Security Crawler will perform a top-level scan of a website and obtain as many details as it can regarding any vulnerabilities the site may have without actually performing a penetration test.  It will then store these details in a database and provide a notification to the system's administrator of the successful crawl.\\\\
This document captures the various requirements and specifications of the Security Crawler system.  It is grouped into details on the background of the project, its stakeholders, its features, use cases that reflect the various capabilities of the project and the features they map to, the various functional and non-functional requirements of the system, and constraints the system must conform to.\\\\
Additionally, there are details on the coding standards, change control methods, and test cases that apply to the implementation and testing of the system.\\\\
Because a usability study of the Security Crawler system would be trivial, a usability study of iOS functionality has been included in this document.  Functions tested include creating and modifying reminders, receiving and responding to text messages, checking weather via the pulldown menu, and creating and modifying events on the calendar.